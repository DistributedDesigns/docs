\section{Automated testing}
Testing the code base was broken up into automated testing and manual testing.
Since day trading systems deal with real currency and software errors could mean the loss of millions of dollars, automated unit testing of critical code was essential to ensure no financial losses in the system.

\subsection{Unit tests}
The currency module of the project was hand rolled rather than importing an existing currency codebase.
While this gave us complete control over functionality and allowed for advanced custom functionality, it meant that thorough testing was required to ensure no floating point, overflow, or other errors that could cause financial bugs.
Unit tests were written to ensure complete code coverage over the currency repository, including all conversion and arithmetic functions.
Due to currency having 100\% code coverage with unit testing meant that all bugs that were encountered throughout development that involved financial sums could be immediately narrowed down to being the responsibility of the business logic of the worker.
This significantly reduced development time as bugs were simple to track down and the root causes were quickly apparent.

\subsection{Travis CI}
Travis CI was also used for automated testing.
We followed the standard industry process for introducing new code into production; that being creating a branch with changes, and creating a pull request before merging into master.
Travis was run on all pull requests, and a pass was required before a merge could happen.
Travis ran two primary checks on all code; a check to see if the code passed a linter, and a check to see that it built with no errors.
The build checks were very simple, Travis simply attempted to compile the code and would report any failures and errors automatically and would report them back to the development team.
The linter that was run was the gometalinter by Alec Thomas, and served as the defacto style guide for the entire codebase.
To prevent linter errors, all team members also installed the linter on their development machines to help ensure no stylistic errors would make it to production.
