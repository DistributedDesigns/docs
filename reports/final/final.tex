%/////////////////////////////////////////////////////////%
%//						PREAMBLE						//%
%/////////////////////////////////////////////////////////%

\documentclass[11pt]{report}

%%%%%%%%%%%%%%%%%%%%%%%
% 	  Packages
%%%%%%%%%%%%%%%%%%%%%%%

%% Fonts and Symbols
%% --------------------------
\usepackage{
			amsmath,			% math operators
			amssymb,			% math symbols
			}

%% Graphics
%% --------------------
\usepackage[pdftex,dvipsnames]{xcolor}  % Coloured text etc.
\usepackage{
			graphicx,			% allows insertion of images
			subfigure,			% allows subfigures (a), (b), etc.
			}
\graphicspath{ {graphics/} }	% (graphicx) relative path to graphics folder

%% Tables
%% --------------------------
\usepackage{
			booktabs,			% better tables, discourages vertical rulings
			multicol,			% allow multi columns
      multirow,     % allow multi rows
			}

%% Layout Alteration
%% --------------------------
\usepackage{
			enumitem,			% indented items for glossary
			framed,				% nice boxes; used in Supervisor's Approval
			geometry,			% change the margins for specific PAGES
			}
\setlist[description]{style=nextline}
\geometry{						% specify page size options for (geometry)
			a4paper, 			% paper size
			margin=1in,			% specified independently with hmargin vmargin
		 }

%% Units
%% --------------------------
\usepackage{
			siunitx,			% has S (decimal align) column type
			}
\sisetup{input-symbols = { () },  % do not treat "(" and ")" in any special way
         group-digits  = false, % no grouping of digits
%		 load-configurations = abbreviations,
%		 per-mode = symbol,
		 }

%% Misc
%% --------------------------
\usepackage{
			pdfpages,			% import pdfs into this document -- used for the Letter
			url,				% allows urls to be added to document
      xargs,      % more than one optional arg in new commands
			}
% preserve default font for URLs
\renewcommand*{\UrlFont}{\rmfamily}
% todo notes setup
\usepackage[colorinlistoftodos,prependcaption,textsize=small]{todonotes}

%% Bibliography
%% --------------------------
\usepackage[
	backend=biber,
	style=ieee]
{biblatex}
\addbibresource{final.bib}


%%%%%%%%%%%%%%%%%%%%%%%
% 	Environments
%%%%%%%%%%%%%%%%%%%%%%%
% Hides the formatting for the summary
\newenvironment{Overview}
	{ % beginning formatting
		% manually add entry to the toc since section*
		% suppresses addition to toc
		\addcontentsline{toc}{chapter}{Overview}
		\topskip0pt				% remove top padding
		\vspace*{\stretch{2}}	% Pad 2/3 of the page length
		\chapter*{Overview}		% don't append a chapter num before "Overview"
	}
	{ % end formatting
		\vspace*{\stretch{3}}
	}


%%%%%%%%%%%%%%%%%%%%%%%
% Macros and Commands
%%%%%%%%%%%%%%%%%%%%%%%

% todo note styles
\newcommandx{\stub}[1]{\todo[linecolor=OliveGreen,backgroundcolor=OliveGreen!25,bordercolor=OliveGreen,inline=true]{#1}}
\newcommandx{\maybe}[1]{\todo[linecolor=blue,backgroundcolor=blue!25,bordercolor=blue,inline=true]{#1}}
\newcommandx{\improvement}[2][1=]{\todo[linecolor=Plum,backgroundcolor=Plum!25,bordercolor=Plum,#1]{#2}}
\newcommand{\citeneeded}{\todo[linecolor=red,backgroundcolor=red!25,bordercolor=red]{cite needed}}
\newcommandx{\thiswillnotshow}[2][1=]{\todo[disable,#1]{#2}} % demo: add 'disable' to note types

% override S column type with centered text column
\providecommand{\textcol}[1]{\multicolumn{1}{c}{#1}}

% provides a place to write on documents; like __________________ that
\providecommand{\blankline}[1]{\rule{#1}{0.5pt}}

% Set up page numbering for appendices as (Appendix Letter) - (Page Number)
\providecommand{\StartAppendices}{
    \newpage
    \newcounter{AppendixCounter}
    \renewcommand{\thepage}{\Alph{AppendixCounter} \textendash\ \arabic{page}}
}

% Manually construct the section title for each appendix and then
% add an entry to the ToC
\providecommand{\Appendix}[1]{
    \newpage
    \stepcounter{AppendixCounter}
    \setcounter{page}{1}
    \section*{Appendix \Alph{AppendixCounter}\quad #1}
    \addtocontents{toc}{\protect\contentsline{chapter}%
    	{Appendix \Alph{AppendixCounter}\quad #1}{}}
	% \protect preserves the spacing in the ToC
}

% Blank lines for marks on the title page
\providecommand{\markline}{\rule{1cm}{0.5pt}}

%/////////////////////////////////////////////////////////%
%//						BODY							//%
%/////////////////////////////////////////////////////////%

\begin{document}

%%%%%%%%%%%%%%%%%%%%%%%
% 	  Title Page
%%%%%%%%%%%%%%%%%%%%%%%
\pagenumbering{gobble}		% turn off page numbering
\begin{titlepage}

  \begin{center}
    \begin{LARGE}
      Department of Electrical and Computer Engineering \\
      University of Victoria \\
      SENG 462 \textemdash{} Distributed Systems and the Internet \\[1cm]
      \textsc{Project Report}
      \\[1in]
    \end{LARGE}
  \end{center}

  \begin{tabular}{ p{0.25\textwidth} p{0.75\textwidth} }
    Report submitted on:& 11 April, 2017 \\
    To: & Prof. S. Neville \\
    & \\
    Names: & J. Cooper (V00XXXXXX)\\
    & T. Stephen (V00812021)\\
    & J. Vlieg (V00XXXXXX)\\[1in]
  \end{tabular}

  \begin{center}
    \begin{tabular}{rl}
      Architecture and project plan & \markline{} /5 \\
      Security & \markline{} /5 \\
      Test plan & \markline{} /5 \\
      Fault tolerance & \markline{} /5 \\
      Performance analysis & \markline{} /5 \\
      Capacity planning & \markline{} /5 \\
      & \\
      \textbf{Total} & \markline{} \textbf{/30} \\
    \end{tabular}
  \end{center}

\end{titlepage}


%%%%%%%%%%%%%%%%%%%%%%%
%    Front matter
%%%%%%%%%%%%%%%%%%%%%%%
\newpage
\addtocontents{toc}{~\hfill\textbf{Page}\par}	% 'Page' above the page numbers
\tableofcontents

\newpage
\pagenumbering{roman}	% i, ii, iii, ... page numbering
\addcontentsline{toc}{chapter}{\listfigurename}	% manually add to toc
\addtocontents{lof}{~\hfill\textbf{Page}\par}
\listoffigures

\newpage				% LoF and LoT may be on the same page if they fit
\addcontentsline{toc}{chapter}{\listtablename}
\addtocontents{lot}{~\hfill\textbf{Page}\par}
\listoftables

\begin{Overview}

The goal of the project was to build a distributed day trading system, with a focus on performance. The client original stated the following actions as business requirements:

\begin{itemize}
  \item View their account
  \item Add money to their account
  \item Get a stock Quote
  \item Buy a number of shares in a stock
  \item Sell a number of shares in a stock they own
  \item Set an automated sell point for a stock
  \item Set a automated but point for a stock
  \item Review their complete list of transactions
  \item Cancel a specified transaction prior to its being committed
  \item Commit a transaction
\end{itemize} 

Additionally, the overall architectural goals of the system were:
\being{itemize}
  \item Minimum transaction processing times. 
  \item Full support for required features. 
  \item Reliability and maintainability of the system. 
  \item High availability and fault recoverability (i.e. proper use of fault tolerance) 
  \item Minimal costs (development, hardware, maintenance, etc.) 
  \item Clean design that is easily understandable and maintainable. 
  \item Appropriate security 
  \item A clean web-client interface. 
\end{itemize}

These requirements form the basis for the distributed daytrading system commissioned by DayTrading Inc.


\end{Overview}


%%%%%%%%%%%%%%%%%%%%%%%
%		Main Body
%%%%%%%%%%%%%%%%%%%%%%%
\newpage
\pagenumbering{arabic}	% 1, 2, 3, ... page numbering

\chapter{Capacity planning}
Progressing through the series of workload files stresses different parts of the trading system architecture.
Workload files differ by the number of unique users, quotes and total transactions.
Early workload files have few unique users and quotes but rapidly increase the number of transactions.
The primary design goal becomes minimizing individual transaction times.
Later workloads rapidly increase the number of unique users and quotes, exposing inefficiencies in different parts of the system that were often the result of optimizing a design for high transaction throughput for early workloads.

This chapter outlines the often surprising manner in which our system failed as we progressed through the workloads, and the design changes that followed.
Most often, the manner of failure was the result of an assumption that would be true at smaller scales but became invalid at a later point.

\section{Technology}

\subsection{Golang}
Golang was selected originally because it gives a lot of tools to maximize scalability, but is also very easy to learn and use efficiently.
Golang’s idioms are relatively easy to learn and the architectural decisions Rob Pike made when bringing the language to fruition heavily favor the task at hand.

Golang has extremely lightweight threading.
It is possible to spawn thousands of threads in a golang instance without bogging down a system.
New threads are made for goroutines, http request, etc, and are managed by the golang thread scheduler, not the OS.
This allows for a lot of multithreaded benefit without a lot of the multithreaded headache.
Golang goroutines and the producer/consumer model are very easily implemented.
We wanted to utilize a language which was able to:

\begin{itemize}
  \item Efficiently scale
  \item Work well with RMQ
  \item Be easy to learn and fun to use.
\end{itemize}

Golang fulfills all of these categories.
The idea of channels in golang is almost identical to the consumption of RabbitMQ channels, and was very easy to hook into the golang flow.

Golang has the added benefit of being both type safe, and well supported by the community.
The former makes it incredibly easy to debug and refactor code.
The latter means that there are a whole set of profiling and testing tools available by default, so we did not have to go out and research what the best ones were.

\subsection{RabbitMQ}
We selected RabbitMQ early on as we wanted to do an event based system, and wanted to be able to communicate between components effortlessly.
With an event based system, we wanted to utilize a communication protocol that would perform FIFO communication between microservices.
Rabbit fills this need exactly, and the documentation is verbose enough that ramp up time is minimal.
It is easily deployed with docker, and the management console is very easy to use and get a good picture of the system state.
It also offers the ability to inject messages directly into queues and exchanges, as well as many other testing features.

\subsection{Redis}
Redis is a distributed, in memory key-value store that supports publish-subscribe events.
We used Redis to provide caching for quotes, unhandled transactions and as a log message buffer.
We used key TTLs to enforce quote validity windows.
Although Redis operates as an in memory data store it writes changes to disk every sixty seconds in order to minimize data loss on system failure.

\subsection{Postgres}
Changes to user account state, retrieved quotes and user transactions are logged in a Postgres database for storage.
Postgres offers full ACID compliance which allows the database to act as a ``source of fact'' for recreating the historical state of accounts in the system.

\subsection{Websockets} 
Websockets allow asynchronous communication between the frontend and backend system components.
Data can be passed to a webpage without requiring a page reload or other user-triggered action.
This allows asynchronous events, like pending buy expirations or the fulfillment of a sell trigger, to be brought to the user's attention immediately.

\section{Original architecture}
\textit{Can steal most of this from the first report.}

\section{Technology}

\subsection{Golang}

\subsection{RabbitMQ}

\subsection{Redis}

\subsection{Postgres}

\subsection{Websockets}
\section{Work plan}
We held many meetings throughout the term on the subject of planning and task division. On average, we would meet once every week or two to modify design decisions and evolve the architecture. Whiteboarding of the design and architecture was documented.

\subsection{Timeline}
Cumulatively, we logged over 450 hours on this prototype. Figure~\ref{fig:work-per-week} shows the week-by-week breakdown. On average, we spent 38.125 hours building the prototype. 

\begin{figure}[tbph]
  \centering
  \includegraphics[width=0.85\linewidth]{graphics/work-per-week}
  \caption{Work per week on prototype, all tasks}
  \label{fig:work-per-week}
\end{figure}

Figure~\ref{fig:work-pie} shows that over half of the work time was spent on implementation. Design sessions took the form of collaborative whiteboarding and brainstorming activities with the group as a whole. 

\begin{figure}[tbph]
  \centering
  \includegraphics[width=0.7\linewidth]{graphics/work-pie}
  \caption{Areas of work}
  \label{fig:work-pie}
\end{figure}

System testing was almost exclusively through code review, where implementations were examined by other group members for output validity and code quality.



\section{Final architecture}
The final architecture, shown in Figure~\ref{fig:arch-final}, was as a whole relatively similar to our planned architecture. 

\begin{figure}[tbph]
  \centering
  \includegraphics[width=0.85\linewidth]{graphics/arch-final}
  \caption[Final system architecture]{Final system architecture. Red databases are Redis caches, blue databases are Postgres.}
  \label{fig:arch-final}
\end{figure}

\subsection{Incoming traffic}
The method of load balancing had to be modified to account for serving of the frontend and connection of the websockets.
Request which enter the load balancer are given a session token, and are always routed back to the same worker.
This allows us to skip the microservice which would be responsible for serving an ACID database of userdata.
The root url serves an HTML page, which contains the login/create page if a user has not registered.
Once at this page, a user is able to register/login, at which point a socket connection will be made to the backend.
This socket connection will serve to asynchronously return the account state, as well as any output messages that need to be presented to the user.
We opted for an entirely event based pattern instead of trying to modify a request/response style page loading to fit the event system on the backend.
This allows all of our requests to be non blocking, and for the user to have the most up to date information all the time.

\subsection{Frontend}
The frontend consists of an authentication screen, and a loggedin view.
Each action results in a post request getting sent to the API.
Login maps to the \texttt{/login} API endpoint, create maps to the \texttt{/create} API endpoint, and all other actions map to the \texttt{/push} API endpoint.
Login and create are responsible for user authentication.
Once this is completed, the frontend also requests a consistent websocket connection over the \texttt{/ws} API endpoint.
This websocket connection is added to the socket hub on the backend.
Any time an event is completed on the backend, the socket hub returns the user’s socket connection and the event is fired over the socket to the users frontend.
Each response contains two portions of data: The updated user state, and a message to display on the output screen.
The updated user state is parsed on the frontend and displayed to the user.

\subsection{Worker}
The worker as a whole functioned relatively similar to the intended architecture.
It stores account data, a local quote cache, and is the workhorse of the entire distributed system.

\subsubsection{Account store}
Accounts are stored on the worker in memory.
Each worker is responsible for a sectioning of accounts.
Once a user has created an account on a worker, all of it’s incoming requests will be routed through that worker until the account is terminated.
While an in memory solution may not seem fault tolerant, all of the events are logged to an audit server, which can be used to replay events on another worker should the original worker crash or go down.
In the case of total system failure, the audit logger uses a persistent storage, Postgres, and the system could be brought online simply be replaying all of the transactions from when the audit logger was last live, to when it last went dark.

\subsubsection{Worker goroutines}
The worker ended up divided into seven goroutines.
Golang goroutines can be thought of simply as threads or processes which are time sliced.

\paragraph{\texttt{incomingTxWatcher}}
This was responsible for the http setup of the websocket handshaking, the frontend serving, as well as the \texttt{/push} API spinup.

\paragraph{\texttt{sendAutoTx}}
This goroutine spins up and pulls from two channels: \texttt{autoTxInitChan} and \texttt{autoTxCancelChan}.
When a user correctly produces an auto transaction by setting and amount and then a trigger value, it is pushed into the channel and pulled by this go routine.
In the case of an \texttt{autoTxInit}, we verify the local quote cache to determine if we can fire the trigger instantly, without the help of the autoTx manager.

If the trigger is valid already, it is fulfilled without leaving the worker.
If not, we simply publish it to the autoTx manager through its initialization autoTx RMQ.

\texttt{AutoTxCancel} works similarly, except any cancel requests are simply sent straight to the worker.
No trigger check is done, since there is no trigger check to do for a cancel.

\paragraph{\texttt{receiveAutoTx}}
This goroutine is the sister routine to \texttt{sendAutoTx}. \texttt{AutoTxFilled} types are sent back to the workers \texttt{autoTxQueue}, which is an RMQ direct exchange where the routing key is the worker ID.
When these filled requests come in, they contain an \texttt{AutoTxKey}, an \texttt{AddStock}, and an \texttt{AddFunds}.
The idea of this is that both buys and sells can be consumed on the same type, as the result of a buy is an amount of stock, and a remainder of cash, and the result of sell is an amount of stock (0) and a remainder of cash.
By unifying these two types, its possible to eliminate the forking behavior that would exist otherwise.

\paragraph{\texttt{catchQuoteBroadcasts}}
This goroutine is relatively simple, as it simply consumes from the quote broadcast exchange and caches the quote into the workers quote cache.

\paragraph{\texttt{fetchNewTx}}
This goroutine is responsible for consuming operations out of the the workers Redis queue and pushing them into the \texttt{unprocessedTx} channel, which is then pulled from by the \texttt{txWorker}.

\paragraph{\texttt{txWorker}}
This goroutine is responsible for the brunt of the processing which occurs on the worker.
Whenever an unprocessed transaction is pushed into the \texttt{unprocessedTx} channel, it is then processed by this worker.
The command is then parsed into a command object, which contains an execute function based on which type of transaction it is.
For example, add transactions will execute the add command in the backend.

\paragraph{\texttt{cleanAccountStore}}
Wakes every sixty seconds to remove expired buys and sells from all user accounts.
This prevents the memory loss associated with a user who initiates a large amount of buys or sells but never confirms those actions.

\subsection{Quote manager}
Quote updates involve every part of the distributed system and are the best demonstration of the efficiency of event-sourced architecture.
With event-sourcing, each service is free to define its own actions to events that occur in other parts of the system.
Figure~\ref{fig:arch-quotes} shows a typical quote request process.

\begin{figure}[tbph]
  \centering
  \includegraphics[width=0.85\linewidth]{graphics/arch-quotes}
  \caption{Service interactions for fulfilling a quote}
  \label{fig:arch-quotes}
\end{figure}

Workers generate requests for new quotes, specifying a stock and whether a cached response is allowed. (A \textsc{Buy} for a stock whose quote is about to expire can force a the retrieval of a fresh quote instead of passing the stale quote to the user.) The Quote manager pulls requests from \texttt{quote\_req} RMQ queue and checks for a cached value.
On a cache miss, or when forced by a worker, a new quote is requested from the legacy service using the timeout procedure detailed in~\ref{sec:timeout-effectiveness}.

Quotes are broadcast with a key consisting of the stock name and whether the quote was retrieved from a cache or from the legacy service (referred to as ``cached'' and ``fresh'' respectively).
Each part of the system filters broadcast messages for its particular purpose.
Workers and AutoTX managers need to maintain a local quote cache so they filter ``fresh'' quotes for cache updates.
The Audit Logger records quote requests for billing with the same filter.
The worker that generated the original request filters for the stock name.
This allows two workers to request and block on updates for the same quote and race their requests.
Both workers would resume execution when the first quote resolves.


\subsection{Audit logger}
The audit logger required significant redesign in order to accommodate high throughput logging.
Section~\ref{sec:log-buf} gives a thorough overview of the necessity of this design and its limitations.

\subsection{AutoTX manager}
The auto transaction manager was designed to be a central point where transactions could be confirmed and sent back to the respective workers. \texttt{AutoTxInit} messages are sent from the worker containing an \texttt{AutoTxKey} (\texttt{Stock}, \texttt{UserID}, \texttt{Action}) as well as an amount and a trigger. \texttt{AutoTxInit} objects are taken and inserted into an AutoTxStore, which contains a map of all \texttt{autoTxKeys} to \texttt{autoTxInits} (For cancellation of transactions), and a map of \texttt{TreeKeys} (\texttt{Stock}, \texttt{Action}) which maps a \texttt{Stock} and \texttt{Action} to a \texttt{Tree}.
When multiple auto transaction buys for a stock ABC comes in, they are inserted into a left leaning red black tree.
A red black tree was chosen because it excels at heavy read/write workloads and it is self balancing.
The auto transaction manager is responsible for requesting quote updates for each stock which resides in it’s store.

The auto transaction manager is also subscribed to the same quote broadcast exchange as the other workers.
When a new quote comes in, the buy and sell trees for that stock are observed.
It is very simple to partition the tree into fillable transactions and unfillable transactions, because red black trees are binary search trees and are balanced and ordered by their very nature.
For each fired trigger, the node is removed from both the tree and the map and then the \texttt{AutoTxFilled} transaction is sent back to the worker which instantiated it.

When \texttt{autoTxCancels} arrive at the autoTx manager, they are simply removed from the autoTx store.

While the auto transaction manager is the only place where user data meets, user data will never interact with any other user data, as comparisons are only done between the quote value and the amount specified by the user.
This ensures no cross contamination of data, or mismatching of auto transactions.


\section{Further improvements}
We were able to create a prototype system that met all of the client requirements before the deadline.
If our prototype is chosen for further development there are some obvious next steps that would begin implementation immediately.

\subsection{Frontend aesthetics}
While the current frontend is functional (Figure~\ref{fig:fe}), it is difficult to use and not aesthetically pleasing. The user experience design is non-existent, and while it fulfills the business logic, it does not function in a way that allows advanced users to capitalize on the structure of the frontend to perform actions more efficiently.

\begin{figure}[tbph]
  \centering
  \includegraphics[width=0.7\linewidth]{graphics/fe}
  \caption{Current frontend design}
  \label{fig:fe}
\end{figure}

\subsection{Encryption}
While the websockets are a one-to-one connection between the user, the information is not currently encrypted in any sort of way that would prevent a Man In The Middle (MITM) attack. In the very near future, we would enjoy performing end-to-end encryption to protect the user's information as it transitions through the websphere.

\subsection{Social media authentication}
While our current authentication system allows much freedom for management on the backend, it is but another “roll your own” authentication system that the user must create an account for. By adding social media authentication such as Facebook or Google Oauth, we would decrease the barrier of entry for new users to enter the platform. By doing this we could grow the larger user numbers, and work on the scalability of the system if users were to near tens or hundreds of thousands. Facebook currently has almost 2 billion users, and allowing these users access to our system in one click is an incredible way to promote user growth.  Additionally the current authentication system stores user information in the browser’s localstorage.  A competent user could modify their localstorage account info to that of another user and gain unauthorized access to their account.  This security hole would be eliminated by the use of social media authentication.

\subsection{Higher TPS}
Additional TPS can be achieved by adding new workers. A thorough discussion of the limits and methods for scaling is in~\ref{sec:scaling}.

\subsection{Mesos or Kubernetes deploy}
Docker proved to be relatively useful for setup/teardown of RMQ, Redis and other services. However, any time we wanted to do a VM deploy, we would be forced to manually run each docker instance and set everything up. Further improvements would include the addition of Apache Mesos or Google Kubernetes for managing Docker deployments. This would cut down significantly on setup time for full scale runs, as well as allow simple runtime tweaking for different environments such as production or testing.


\chapter{Capacity planning}
Progressing through the series of workload files stresses different parts of the trading system architecture.
Workload files differ by the number of unique users, quotes and total transactions.
Early workload files have few unique users and quotes but rapidly increase the number of transactions.
The primary design goal becomes minimizing individual transaction times.
Later workloads rapidly increase the number of unique users and quotes, exposing inefficiencies in different parts of the system that were often the result of optimizing a design for high transaction throughput for early workloads.

This chapter outlines the often surprising manner in which our system failed as we progressed through the workloads, and the design changes that followed.
Most often, the manner of failure was the result of an assumption that would be true at smaller scales but became invalid at a later point.


\chapter{Capacity planning}
Progressing through the series of workload files stresses different parts of the trading system architecture.
Workload files differ by the number of unique users, quotes and total transactions.
Early workload files have few unique users and quotes but rapidly increase the number of transactions.
The primary design goal becomes minimizing individual transaction times.
Later workloads rapidly increase the number of unique users and quotes, exposing inefficiencies in different parts of the system that were often the result of optimizing a design for high transaction throughput for early workloads.

This chapter outlines the often surprising manner in which our system failed as we progressed through the workloads, and the design changes that followed.
Most often, the manner of failure was the result of an assumption that would be true at smaller scales but became invalid at a later point.

\section{Automated testing}
Testing the code base was broken up into automated testing and manual testing.  Since day trading systems deal with real currency and software errors could mean the loss of millions of dollars, automated unit testing of critical code was essential to ensure no financial losses in the system.

\subsection{Unit tests}
The currency module of the project was hand rolled rather than importing an existing currency codebase.  While this gave us complete control over functionality and allowed for advanced custom functionality, it meant that thorough testing was required to ensure no floating point, overflow, or other errors that could cause financial bugs.  Unit tests were written to ensure complete code coverage over the currency repository, including all conversion and arithmetic functions.  Due to currency having 100\% code coverage with unit testing meant that all bugs that were encountered throughout development that involved financial sums could be immediately narrowed down to being the responsibility of the business logic of the worker.  This significantly reduced development time as bugs were simple to track down and the root causes were quickly apparent.

\subsection{Travis CI}
Travis CI was also used for automated testing.  We followed the standard industry process for introducing new code into production; that being creating a branch with changes, and creating a pull request before merging into master.  Travis was run on all pull requests, and a pass was required before a merge could happen.  Travis ran two primary checks on all code; a check to see if the code passed a linter, and a check to see that it built with no errors.  The build checks were very simple, Travis simply attempted to compile the code and would report any failures and errors automatically and would report them back to the development team.  The linter that was run was the gometalinter by Alec Thomas, and served as the defacto style guide for the entire codebase.  To prevent linter errors, all team members also installed the linter on their development machines to help ensure no stylistic errors would make it to production.
\section{Manual testing}
The majority of testing time was spent on extensive manual testing.  All pull requests were manually tested at least twice, once by the author, once by a reviewer.  Manual testing primarily focused on ensuring that business logic was upheld, and that bad inputs were handled correctly.  Since Rabbit was used for the queue that the worker would pull all of its data and workload from, testing was simply a matter of pushing test cases to the queue.  Sample workload files were created to run the worker against, and the output of the worker would then be compared against the correct output file.

The inclusion of an extensive hand rolled logging library into the codebase also made manual testing significantly easier; console output could be broken down into multiple levels such as debug, critical, info and others.  This allowed for very simple debugging as debug statements could be embedded throughout the code and suppressing them at run time was simply a matter of increasing the displayed log level.

Another factor in simplifying our testing was the fake quote server.  Complete control over what quotes were returned allowed for consistent reproduction of bugs, as well as creation of specific test scenarios that could break or stress test our product.

Many test scenarios were run at all stages of development.  Primary focus was on ensuring the critical business functions all performed as expected and according to the SLA.  Specific workload files to test various buy, sell, and auto transaction cases were created and run on the system multiple times to ensure no failures or SLA violations.

\chapter{Capacity planning}
Progressing through the series of workload files stresses different parts of the trading system architecture.
Workload files differ by the number of unique users, quotes and total transactions.
Early workload files have few unique users and quotes but rapidly increase the number of transactions.
The primary design goal becomes minimizing individual transaction times.
Later workloads rapidly increase the number of unique users and quotes, exposing inefficiencies in different parts of the system that were often the result of optimizing a design for high transaction throughput for early workloads.

This chapter outlines the often surprising manner in which our system failed as we progressed through the workloads, and the design changes that followed.
Most often, the manner of failure was the result of an assumption that would be true at smaller scales but became invalid at a later point.


\chapter{Capacity planning}
Progressing through the series of workload files stresses different parts of the trading system architecture.
Workload files differ by the number of unique users, quotes and total transactions.
Early workload files have few unique users and quotes but rapidly increase the number of transactions.
The primary design goal becomes minimizing individual transaction times.
Later workloads rapidly increase the number of unique users and quotes, exposing inefficiencies in different parts of the system that were often the result of optimizing a design for high transaction throughput for early workloads.

This chapter outlines the often surprising manner in which our system failed as we progressed through the workloads, and the design changes that followed.
Most often, the manner of failure was the result of an assumption that would be true at smaller scales but became invalid at a later point.

\section{Decreasing quote retrieval time}\label{sec:qs}
The legacy quote server can delay for up to four seconds before sending a response.
This delay is vastly greater than the typical command execution time of dozens of microseconds (see~\ref{sec:cmd-dist}).
The legacy server is the greatest barrier to high command throughput and dozens of hours of research and design was spent mitigating the effects of its delay.

\subsection{Statistical analysis of legacy quote server}
We sent a large number of serial requests to the quote server and recorded the response time with a shell script.
Figure~\ref{fig:legacy-qs-hist} shows the distribution of response times follow an exponential distribution\improvement{do a goodness of fit test}{} with $65.75\%$ of responses experiencing only network delay.
The expected value of the response time is \SI{563.3}{\milli\second}.

\begin{figure}[tbph]
  \centering
  \includegraphics[width=0.6\linewidth]{../../data/quoteserver-times/hist}
  \caption[Legacy quote server response times]{Histogram of legacy quote server response times with \SI{1}{\second} buckets}
  \label{fig:legacy-qs-hist}
\end{figure}

The one second buckets in~\ref{fig:legacy-qs-hist} obscures the fact that results are clustered after whole second values.
Removing the constant delay portion from each bucket yields the distribution (Table~\ref{tbl:qs-net-delay}) of variable network and processing delays.

\begin{table}[htpb]
  \centering
  \caption{Legacy quote server network delays}
  \label{tbl:qs-net-delay}
  \begin{tabular}{@{}cccccc@{}}
    \toprule
    Minimum & 1st Quartile & Median & Mean & 3rd Quartile & Maximum \\ 
    \midrule
    \SI{7}{\milli\second} & \SI{9}{\milli\second} & \SI{9}{\milli\second} & \SI{9.414}{\milli\second} & \SI{10}{\milli\second} & \SI{26}{\milli\second} \\ 
    \bottomrule
  \end{tabular}
\end{table}

From this data we can conclude that if the legacy quote server has not sent a response after \SI{30}{\milli\second} then we will wait at least \SI{1}{\second} for a response.

\subsection{Using timeouts to ensure fast quote retrieval}\label{sec:qs-timeout}
We decided to use a request timeout strategy to minimize the total time spent waiting for a quote.
If there was no response from the quote server after a given timeout we cancel the request by closing the socket connection and issue a new request.
We used the tail of the network delay data (i.e.\ ..., 16, 16, 17, 17, 20, \SI{26}{\milli\second}) to set an initial timeout at \SI{20}{\milli\second} and used a \SI{5}{\milli\second} exponential backoff.
This backoff strategy requires six iterations to exceed the expected delay of \SI{563.3}{\milli\second}.
Exceeding \SI{1}{\second} total delay has a likelihood of 0.0299\%.
If the total timeout exceeds \SI{4}{\second} and there is still no response from the quote server then the service is assumed unreachable and the quote manager raises an error.

\subsection{Timeout effectiveness}
We implemented the timeout strategy in~\ref{sec:qs-timeout} and gathered response time data directly from the quote manager.
Figure~\ref{fig:retries} shows frequency of retry attempts before a quote resolved.
Strangely, the distribution does not match that of Figure~\ref{fig:legacy-qs-hist} and the legacy quote server has an 89.35\% likelihood (instead of 65.75\%) of resolving in under \SI{20}{\milli\second}.
Day Trading Inc.\ has assured us that the behavior of the legacy quote server is stationary so perhaps the change arises from our serial and timeout-retry request methods.

\begin{figure}[tbph]
  \centering
  \includegraphics[width=0.6\linewidth]{../../data/quote-times-verification/retries}
  \caption{Quote retry attempt histogram}
  \label{fig:retries}
\end{figure}

The distribution of total waiting times is shown in Figure~\ref{fig:waiting-time}. 89.18\% of quotes are resolved before \SI{50}{\milli\second}.

\begin{figure}[tbph]
  \centering
  \includegraphics[width=0.6\linewidth]{../../data/quote-times-verification/waiting-time}
  \caption[Total waiting time to retrieve a quote]{Total waiting time to retrieve a quote with \SI{50}{\milli\second} buckets}
  \label{fig:waiting-time}
\end{figure}

Only three quotes take longer than \SI{500}{\milli\second} and none longer than \SI{850}{\milli\second}.
This adds confidence to our derivation that waiting longer than \SI{1}{\second} for a quote should be a $\approx \frac{1}{3400}$ event.

\subsection{Minimizing lingering TCP connections}
When a TCP connection is terminated with a FIN command the socket enters the TIME-WAIT state until the termination is acknowledged with a FIN-ACK command.
The socket stays in the TIME-WAIT state for twice the connection MSL, typically two minutes.
Each open socket occupies approximately 1 Mb of memory and is considered an open file.
Thus, the number of open sockets is limited the the system memory and OS limitation on concurrent open files.

The legacy quote server timeout method necessarily leaves many connections in the TIME-WAIT state.
When we first implemented the timeout method the quote manager would become unresponsive and crash as connections lingered in the TIME-WAIT state and the host had its memory occupied entirely with open TCP connections.

Changing connection methods in our application from the generic \texttt{Dial} method to the specific \texttt{DialTCP} method resolved the lingering connection issue.
We suspect that \texttt{Dial} discards the connection in such a way that the FIN-ACK is not received by the socket and the OS maintains the connection for the entire double MSL period.
This difference is behavior is undocumented and should be disseminated to developers who use Go in applications with high TCP socket turnover.

\section{Worker scaling}

\subsection{The sixty second golden window}

\subsection{Scaling results}
\section{Command execution time analysis}

\chapter{Capacity planning}
Progressing through the series of workload files stresses different parts of the trading system architecture.
Workload files differ by the number of unique users, quotes and total transactions.
Early workload files have few unique users and quotes but rapidly increase the number of transactions.
The primary design goal becomes minimizing individual transaction times.
Later workloads rapidly increase the number of unique users and quotes, exposing inefficiencies in different parts of the system that were often the result of optimizing a design for high transaction throughput for early workloads.

This chapter outlines the often surprising manner in which our system failed as we progressed through the workloads, and the design changes that followed.
Most often, the manner of failure was the result of an assumption that would be true at smaller scales but became invalid at a later point.

\section{Logging throughput}
The 1000 user workload was the first occasion for our system to operate at nominal TPS for a significant portion of its runtime\improvement{link to image with 100 v 1000 tps}.
Since each transaction needed to write at least one entry to the audit log the volume of log messages was unprecedented.
Controlling the ``firehose'' of log messages was the most significant architectural redesign.
It included several false starts and, ultimately, reached a workable but flawed solution.

\subsection{Limits of logging to a flat file}
From the initial prototype through the 100 user workload, the audit service wrote directly to an \texttt{.xml} file that could be submitted for validation.
Log messages were removed from RMQ and stored in memory for writing by separate threads.
However, running the 1000 user workload exceeded the rate that the audit service could clear messages from RMQ, causing a significant backlog of messages to develop.
As the total message backlog size approached 700k the rate that messages could be exchanged slowed, causing a slowdown in the rest of the system as execution was blocked on message exchange.
Soon after, services would fail as they lost their connection to the RMQ server.

As noted in the ``Production Checklist'' section of the RMQ user guide\footnote{\url{https://www.rabbitmq.com/production-checklist.html}}, performance is heavily tied to available RAM.
As the backlog increases RMQ will begin swapping RAM to disk to ensure persistence.
The IO penalty for writing to disk causes an intense slowdown.
Since the worker services generating the logs are not capable of throttling they eventually push RMQ into resource exhaustion and failure.

Direct to file logging was never intended for production use.
Creating per-user dumplogs would be onerous since there was no direct method for searching or sorting the log file.
Leaving the log file implementation in place for most of the project allowed us to focus development efforts on optimizing the quote manager and implementing the auto transaction service.
Letting RMQ fail illuminated the ``danger zone'' for RMQ on the lab machines.
Different audit logger refactors could be compared for effectiveness by monitoring the RMQ backlog.

\subsection{Logging directly to an RDBMS}\label{sec:log-rdbms}
The first refactor involved inserting logs into Postgres and writing to a file on an as-requested basis.
This solved the problem of creating per-user log files but throughput was significantly worse than writing direct to a file.
With direct to file, the 100 user workload with 100k transactions generated 2k backlogged messages on RMQ.
With the Postgres refactor, the 100 user workload resulted in a 20k message backlog.
No attempts at larger workloads were made after this poor result.

This performance slowdown is not surprising.
The flat file and RDBMS both store the pre-formatted \texttt{.xml} entry for the event.
In addition, the RDBMS stores extra data about the user name, transaction type and creation time to enable queries.
The RDBMS is storing more data than the flat file.
In addition, the RDBMS suffers a performance penalty from indexing data on insertion.
While there are methods to mitigate these problems, such as connection pooling, the performance degredation was extreme enough to justify larger service refactors.

\subsection{Processing logs with ELK}
An RDBMS enables rich querying and enforces data integrity \textemdash{} useful features that are not relevant for storing an append-only log.
Moreover, the indexing that provides those useful features introduces a performance penalty that limits throughput.
Specialized log storage solutions forgo rich indexing in order to maximize write throughput.

The Elasticsearch - Logstash - Kibana (ELK) suite of applications from elastic.co\footnote{\url{https://www.elastic.co}} is a popular distributed log storage method.
Logstash consumes and transforms data for indexing and storage in Elasticsearch.
Kibana is a graphical monitoring suite that provides information about the logging rate and health of the logstash and elasticsearch services.

A prototype was created that deployed the ELK stack in separate Docker containers collocated with the audit logger.
Logstash consumed messages directly from RMQ and sent them to Elasticsearch for indexing and storage.
The prototype had abysmal performance.
With the 45 user workload there was a 7k (out of 10k total) backlog.
Development was abandoned at this point.

The poor performance of the ELK stack was directly related to its resource limitations.
Each part of the ELK stack performs better with more available RAM.
Collocating all services severely limited the available RAM.
Also, ELK requires a non-trivial amount of JVM and OS tuning to provide optimal resource availability.
Although there are guides for this process it was unclear how to apply their recommendations on the tower of abstractions in the production environment: a docker OS on a VM OS on a host OS, each needing their own tuning.

The Elasticsearch scaling guide\footnote{\url{https://www.elastic.co/guide/en/elasticsearch/guide/current/scale.html}} recommends adding more shards (i.e. independent instances which maintain a partition of the data) to increase write throughput.
This throughput solution \textemdash{} a distributed system within a distributed system \textemdash{} directly links scaling to resource demand.
Scaling Elasticsearch would likely reduce the number of systems available to host workers and limit the maximum TPS.
The problem of high log throughput would be solved by removing the ability to create logs at a high rate.

\subsection{Buffered logging}\label{sec:log-buf}
The problem with the RDBMS solution in~\ref{sec:log-rdbms} is fundamentally a mismatch between the production and consumption rate of log messages.
The solution to this problem is to place a buffer between the producer and consumer.

\begin{figure}[tbph]
  \centering
  \includegraphics[width=0.95\linewidth]{graphics/audit}
  \caption{Buffered audit logger}
  \label{fig:audit}
\end{figure}

Figure~\ref{fig:audit} shows how multiple \texttt{Event Catcher} workers remove incoming log messages from RMQ and place them in Redis. \texttt{Log Inserter} workers remove items from Redis and store them in Postgres.
With this design, the 1000 user workload only reached a 20k backlog of messages and had no observed performance degradation.
Although a run would finish in around \SI{45}{\second} it would be upwards of \SI{6}{\minute} to finish insertion into Postgres.
When a message for a dumplog was processed it would display messages had successfully migrated to Postgres but was unaware of those still in Redis.
This is a soft violation of the business requirement that a dumplog should show all transactions proceeding itself.
We believe this is acceptable since, in actual use, there is no clear ``end of work'' to capture.
All log messages will make it into Postgres for querying so the requirement is \textit{eventually} satisfied.

This method has an upper limit to its effectiveness since the Redis buffer could run out of storage space under periods of sustained high TPS.
The boundary of the buffer memory was not encountered during any testing and we cannot speculate about its value.
In order to determine the limits we would need a workload file larger than the final workload.
Alternatively, we could induce a period of sustained high TPS by removing the requirement to re-fetch expired quotes and concatenate existing workloads.

\subsection{Alternate solutions}
When the buffer method in~\ref{sec:log-buf} reaches its limit there are several possible development paths for proceeding forward:

\begin{enumerate}
  \item \textit{Agglomerate messages}: Message traffic can be reduced by combining multiple log events into one message.
Workers would only emit messages for logging at fixed intervals or after a certain number of events (whichever comes first) and reduce the overhead associated with creating, sending and processing RMQ messages.
The optimal message size would have to be determined through experiment.
  \item \textit{RMQ scaling}: RMQ is capable of its own distributed deployment.
Increasing resources available to the message bus would allow more messages to be stored before removal into the buffer.
  \item \textit{Robust message passing}: Apache's Kafka\footnote{\url{https://kafka.apache.org}} provides functionality similar to RMQ but is optimized for message storage and large backlogs.
This allows consumers to operate at different rates and removes the need to process log messages faster.
\end{enumerate}

\section{Quote manager scaling}
The system's TPS grows almost $10\times$ once a full set of quotes has been retrieved.
Further increases to average TPS could be gained by decreasing the time spent fetching quotes.
The methods in~\ref{sec:qs} brought the quote server response time to its lower limit so further efficiency could only come from horizontally scaling the quote managers.
Intuitively, doubling the number of quote managers should decrease the total time to retrieve all quotes by half, provided the workload is split evenly.
If only it were so simple.

\subsection{Building a ``snoopy'' quote manager}
With one week until the final deadline we decided to refactor the quote manager to participate in a multi-quote manager environment.
We ported the ``snoopy caching'' functionality from the worker and audit logger into the quote manager.
The ``snoopy'' quote server could listen to quote broadcasts and update its local cache accordingly.
With this functionality, multiple quote managers could act as workers servicing requests from a single RMQ queue. 

A message header with the ID of the quote manager that serviced the request was added to all quote broadcasts.
The quote manager cache updaters would discard messages that originated from its own quote manager.
This is inefficient but necessary because RMQ does not allow an ``anti-match'' for message routing keys.
That is, one cannot specify, "Capture all messages \textit{except} ones that follow this pattern.''

Total development effort was approximately one hour.

\subsection{Performance analysis}
Figure~\ref{fig:snoopy-qs} shows that TPS correlates \textit{negatively} with the number of snoopy quote managers.
This is very counter-intuitive and deserves reflection.

\begin{figure}[tbph]
  \centering
  \includegraphics[width=0.85\linewidth]{../../data/tps/multi_vs_single_qs}
  \caption[Snoopy quote manager performance]{Performance of multi and single quote manager designs on 1000 user workload}
  \label{fig:snoopy-qs}
\end{figure}

Comparing the single quote manager deployment in multi and single architectures gives a sense of the overhead associated with discarding quote broadcasts.
The relation becomes less clear when two and three quote managers are used: the system benefits from having to retrieve fewer quotes per quote manager but adding quotes to the cache also has a delay.
The benefits from horizontal scaling start to manifest when three quote managers are used, but it takes the form of, ``things stopped getting worse,'' instead of ``performed better than a single quote manager.''

The single architecture quote manager is very fast because it already functions like a multi machine quote manager.
Each new request spawns a thread that handles communication with the legacy service.
Most of the threads are blocking on a response from the legacy service making it very likely that a new thread will find the application in an idle state.
Since workers block on the completion of a quote command the maximum number of simultaneous quote requests is equal to the number of workers.
Hence, the number of threads requesting quotes in the quote manager is capped, thus preventing thread creation runaways and CPU starvation in the quote manager.
Adding more quote managers doesn't increase the number of simultaneous quotes that can be requested by workers.

\subsection{Alternate solutions}
The snoopy quote manager was implemented because it was a small amount of development effort for a large potential payoff.
There is another way to implement multiple quote managers, although it is more complicated: quote symbols could be hashed to associate with unique quote managers.
This prevents the need for quote managers to do snoopy caching.
However, it introduces problems with scaling since the hash will depend on the number of available quote managers.

The snoopy quote manager can scale easier since failures would be independent.
As the number of workers continues to grow, Figure~\ref{fig:snoopy-qs} should be re-run to determine the break point between the communication overhead and the increased capacity for simultaneous quote requests.


\section{Worker loading}
To run the workload files we loaded sections of 3300 transactions into workers in a round-robin manner.
By measuring the total number of transaction in a worker's backlog at each loading cycle we could determine the limits of our round-robin loading method.

\subsection{Worker backlog analysis}

\begin{figure}[tbph]
  \centering
  \includegraphics[width=0.9\linewidth]{graphics/backlog_by_workers}
  \caption{Total worker backlog for 1000 user workload}
  \label{fig:backlog-total}
\end{figure}

The curves in Figure~\ref{fig:backlog-total} have two distinct parts: an upward slope that becomes steeper as the number of workers before coming to a maximum and either plateauing, as with eight workers; decreasing at a fixed rate, as with nine to eleven workers, or; decreasing and leveling out, as with twelve workers.
It's important to note that the x-axis represents loading cycles and not a uniform time scale.
Round robin loading cycle times increase as the number of workers increases.

The slope of the curve represents the ratio of transactions coming in to a worker over transactions completed between successive cycles.
Essentially, this is
\begin{equation*}
  {\SI{3300}{transactions per cycle} \times \si{cycles per second} \over \si{transactions per second}}
\end{equation*}
The peak in the curves occurs when the system has retrieved a full load of quotes the transactions per second increases drastically.
For eight workers, the slope is approximately 1, indicating that the input and output rates are equal.
The work in~\ref{sec:worker-scaling-results} indicates that the transaction input rate must be approximately \SI{3600}{transactions per second} for each worker.
As the number of workers increases, the number of transactions per second for a worker stays (mostly) constant but the cycles per second decreases.
The leveling off with twelve workers indicates that the workers are starved for transactions near the end of the run.

The change in the rising slopes is also affected by the increased cycle time but the correlation is less direct.
During loading, the transactions per second is significantly lower than 3300 and constant regardless of the number of workers.
The slope increases with the number of workers because the capacity for transactions in the system increases (i.e. each worker has its own cache).
As the cycle times become longer this decreases the slope.

\subsection{Worker starvation analysis}

\begin{figure}[tbph]
  \centering
  \includegraphics[width=0.85\linewidth]{../../data/worker-load/backlog_12w}
  \caption{Workers entering starvation during a 1000 user workload}
  \label{fig:backlog12w}
\end{figure}

Figure~\ref{fig:backlog12w} shows how workers entering starvation at different times.
All workers have identical backlogs up until the peak where higher-numbered workers operate at nominal TPS earlier.
This is because the low-numbered workers early in the round-robin cycle are ``alone'' for longer with the transaction list and are more likely to encounter uncached quotes.
High-numbered workers benefit from the pre-caching.
Unfortunately, they enter starvation approximately ten cycles before loading finishes and represent an inefficient use of resources.

The rates of descent are mostly uniform, with the exception of worker 4 (machine B133).
Consistently, this machine performed worse than its peers.
This could be the result of hardware aging and general, spooky ``cruft'' on the machine or a non-uniformity in the workload distribution.
The latter is unlikely since worker 4 was slow regardless of the total number of workers.

We did not undertake any tests with thirteen workers because of these results with twelve \textemdash{} adding more workers would cause the system to enter starvation earlier and would be a poor use of resources.

\subsection{Alternate solutions}
The ideal operating state is when the incoming and outgoing transaction rates are equal.
In order to achieve this state we would have to implement a feedback system with the transaction loader.
The number of backlogged transactions in a worker could change the number of transactions sent in a loading cycle.

Though not trivial, this is a very tractable solution.
However, we feel it would be overly specific to the testing environment.
Could an actually existing trading system exert backpressure on user loads to throttle demand? This seems unlikely, or at least one that would lead to frustrated users.
Further testing should involve a dynamic \textit{stream} of transactions that could exhibit richer behavior like cyclic demand cycles and surges.
Though the same risk of overfitting the prototype software to the test environment is present, the fidelity has increased and the solutions should be more generally applicable.
 

%%%%%%%%%%%%%%%%%%%%%%%
% 	  Referrences
%%%%%%%%%%%%%%%%%%%%%%%
\newpage
\renewcommand*{\UrlFont}{\rmfamily} % preserve default font for URLs
%\printbibliography[heading=bibintoc,title={References}]

%%%%%%%%%%%%%%%%%%%%%%%
% 	  Back Matter
%%%%%%%%%%%%%%%%%%%%%%%

\StartAppendices{}
\section{Command execution time analysis}

\end{document}
